\documentclass[11pt]{article}

% general packages without options
\usepackage{amsmath,amssymb,bbm}
% graphics
\usepackage{graphicx}
% text formatting
\usepackage[document]{ragged2e}
\usepackage{pagecolor,color}

\newcommand{\noun}[1]{\textsc{#1}}

\usepackage[utf8]{inputenc}
\usepackage[T1]{fontenc}
% geometry
\usepackage[margin=2cm]{geometry}

\usepackage{multicol}
\usepackage{setspace}

\usepackage{natbib}
\setlength{\bibsep}{0.0pt}

%\usepackage[french]{babel}

% layout : use fancyhdr package
%\usepackage{fancyhdr}
%\pagestyle{fancy}

% variable to include comments or not in the compilation ; set to 1 to include
\def \draft {1}


% writing utilities

% comments and responses
%  -> use this comment to ask questions on what other wrote/answer questions with optional arguments (up to 4 answers)
\usepackage{xparse}
\usepackage{ifthen}
\DeclareDocumentCommand{\comment}{m o o o o}
{\ifthenelse{\draft=1}{
    \textcolor{red}{\textbf{C : }#1}
    \IfValueT{#2}{\textcolor{blue}{\textbf{A1 : }#2}}
    \IfValueT{#3}{\textcolor{ForestGreen}{\textbf{A2 : }#3}}
    \IfValueT{#4}{\textcolor{red!50!blue}{\textbf{A3 : }#4}}
    \IfValueT{#5}{\textcolor{Aquamarine}{\textbf{A4 : }#5}}
 }{}
}

% todo
\newcommand{\todo}[1]{
\ifthenelse{\draft=1}{\textcolor{red!50!blue}{\textbf{TODO : \textit{#1}}}}{}
}


\makeatletter


\makeatother







\begin{document}

\title{\vspace{-1cm}Fostering the use of methods for geosimulation models sensitivity analysis and validation
\\\medskip
\textit{ECTQG 2019}
}
\author{\noun{R. Reuillon}$^{1}$, \noun{M. Leclaire}$^{1}$, \noun{J. Raimbault}$^{1,\ast}$, \noun{H. Arduin}$^2$, \noun{P. Chapron}$^3$, \noun{G. Chérel}$^1$,\\
 \noun{E. Delay}$^4$, \noun{F. Lavall{\'e}e}$^5$, \noun{J. Passerat}$^6$, \noun{P. Peigne}$^7$, \noun{J. Perret}$^{1,3}$, \noun{S. Rey-Coyrehourcq}$^8$\medskip\\
$^1$ UPS CNRS 3611 ISC-PIF\\
$^2$ UMR Inserm 1037 CRCT\\
$^3$ Univ. Paris-Est, LaSTIG STRUDEL, IGN, ENSG\\
$^4$ CIRAD UPR GREEN\\%, F-34398 Montpellier, France\\
$^5$ Irstea, LISC \\
$^6$ ConsenSys\\
$^7$ {\'E}cole 42\\
$^8$ UMR CNRS 6266 IDEES
\medskip\\
$^{\ast}$\texttt{juste.raimbault@iscpif.fr}
}
\date{}

\maketitle

\justify

\pagenumbering{gobble}


\textbf{Keywords: }\textit{Validation of simulation models; Multi-modeling; Model calibration; Sensitivity analysis; Incremental modeling; Pedagogy of simulation models}

\medskip

In recent years, there has been a significant increase in the development of methods to explore, validate, calibrate and optimize geosimulation models. These methods and tools remain, however, underused by simulation communities, despite an ever improved and easier access to high performance computation facilities. The OpenMOLE model exploration software~\citep{reuillon2013openmole} is one of the reliable approaches fully dedicated to promote these techniques. This presentation offers some feedback on the recent initiative of a researcher school in model validation, focused around models and practices linked to the OpenMOLE platform. We present the iterative exploration and validation protocol developed during the school, with methods of increasing refinement deployed on a toy geosimulation model (spatialized prey-predator agent-based model of a zombie infection, with multi-modeling paradigms to include diverse processes for agent behavior). First, we illustrate classical sensitivity analysis methods (stochasticity, design of experiments, global sensitivity indices), and then specific methods to study spatial configuration sensitivity, evolutionary computation methods for calibration and diversity search, and Bayesian calibration methods. They are applied on diverse specific submodels, highlighting specific mechanisms of the model, in order to answer associated thematic questions. We also illustrate the comparison with competing model ontologies by calibrating an ODE-based model on data generated by the simulation model. We finally synthesize lessons learned in the final challenge part of the school, consisting of the autonomous exploration of a new model instance by participants, including defining a thematic question and applying appropriate validation methods. This experiment both introduces a broad overview of new geosimulation model methods, and suggests ways to disseminate these into the modeling communities through similar pedagogical implementations.

% HA: is the school an experiment? ^^ JR : yes I think so ; isn't it ?



%%%%%%%%%%%%%%%%%%%%
%% Biblio
%%%%%%%%%%%%%%%%%%%%
%\tiny

%\begin{multicols}{2}

%\setstretch{0.3}
%\setlength{\parskip}{-0.4em}


%\footnotesize

\bibliographystyle{apalike}
\bibliography{biblio}
%\end{multicols}



\end{document}
