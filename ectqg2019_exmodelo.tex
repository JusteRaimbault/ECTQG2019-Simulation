\documentclass[11pt]{article}

% general packages without options
\usepackage{amsmath,amssymb,bbm}
% graphics
\usepackage{graphicx}
% text formatting
\usepackage[document]{ragged2e}
\usepackage{pagecolor,color}

\newcommand{\noun}[1]{\textsc{#1}}

\usepackage[utf8]{inputenc}
\usepackage[T1]{fontenc}
% geometry
\usepackage[margin=2cm]{geometry}

\usepackage{multicol}
\usepackage{setspace}

\usepackage{natbib}
\setlength{\bibsep}{0.0pt}

%\usepackage[french]{babel}

% layout : use fancyhdr package
%\usepackage{fancyhdr}
%\pagestyle{fancy}

% variable to include comments or not in the compilation ; set to 1 to include
\def \draft {1}


% writing utilities

% comments and responses
%  -> use this comment to ask questions on what other wrote/answer questions with optional arguments (up to 4 answers)
\usepackage{xparse}
\usepackage{ifthen}
\DeclareDocumentCommand{\comment}{m o o o o}
{\ifthenelse{\draft=1}{
    \textcolor{red}{\textbf{C : }#1}
    \IfValueT{#2}{\textcolor{blue}{\textbf{A1 : }#2}}
    \IfValueT{#3}{\textcolor{ForestGreen}{\textbf{A2 : }#3}}
    \IfValueT{#4}{\textcolor{red!50!blue}{\textbf{A3 : }#4}}
    \IfValueT{#5}{\textcolor{Aquamarine}{\textbf{A4 : }#5}}
 }{}
}

% todo
\newcommand{\todo}[1]{
\ifthenelse{\draft=1}{\textcolor{red!50!blue}{\textbf{TODO : \textit{#1}}}}{}
}


\makeatletter


\makeatother


\begin{document}







\title{\vspace{-1cm}Fostering the use of methods for geosimulation models sensitivity analysis and validation
\\\medskip
\textit{ECTQG 2019 - Abstract proposal}
}
\author{\noun{R. Reuillon}$^{1}$, \noun{M. Leclaire}$^{1}$, \noun{J. Raimbault}$^{1,\ast}$, \noun{H. Arduin}$^2$, \noun{P. Chapron}$^3$, \noun{G. Cherel}$^1$,\\
 \noun{E. Delay}$^4$, \noun{F. Lavall{\'e}e}$^5$, \noun{J. Passerat}$^6$, \noun{P. Peigne}$^7$, \noun{J. Perret}$^8$, \noun{S. Rey-Coyrehourcq}$^9$\medskip\\
$^1$ UPS CNRS 3611 ISC-PIF\\
$^2$ \\
$^3$ COGIT, IGN\\
$^4$ \\
$^5$ IRSTEA \\
$^6$ \\
$^7$ \\
$^8$ Univ. Paris-Est, LaSTIG STRUDEL, IGN, ENSG\\
$^9$ UMR CNRS 6266 IDEES
\medskip\\
$^{\ast}$\texttt{juste.raimbault@iscpif.fr}
}
\date{}

\maketitle

\justify

\pagenumbering{gobble}


\textbf{Keywords: }\textit{Validation of simulation models; Multi-modeling; Model calibration; Sensitivity analysis; Incremental modeling; Pedagogy of simulation models}

\medskip

Recent years have witnessed a significant development in methods to explore, validate, calibrate and optimize geosimulation models. These methods and tools remain however underused by simulation communities, despite a democratization in the accessibility to high performance computation facilities and of the tools to exploit it such as the OpenMOLE model exploration software~\citep{reuillon2013openmole}. This presentation proposes a feedback on the recent initiative of a researcher school in model validation, focused around models and practices linked to the OpenMOLE platform. We present the iterative exploration and validation protocol developed during the school, with methods of increasing refinement deployed on a toy geosimulation model (spatialized prey-predator model for a zombie infection with multi-modeling paradigms to include diverse processes for agent behavior). We illustrate classical sensitivity analysis methods (stochasticity, experience plans, global sensitivity indices), specific methods to study the sensitivity to spatial configuration, evolutionary computation methods for calibration and diversity search, and Bayesian calibration methods. These are applied on diverse submodels with specific processes activated, in order to answer associated thematic questions. We also illustrate the comparison with competing model ontologies by calibrating a SIR model on data generated by the simulation model. We finally synthesize lessons learnt during the final part of the school consisting in an autonomous exploration by participants of a new model instance unknown to them, including the problematization of a thematic question and the application of appropriated validation methods. This experiment both introduce a broad overview of new model geosimulation methods, and suggests ways to disseminate these into the modeling communities through similar pedagogical implementations.





%%%%%%%%%%%%%%%%%%%%
%% Biblio
%%%%%%%%%%%%%%%%%%%%
%\tiny

%\begin{multicols}{2}

%\setstretch{0.3}
%\setlength{\parskip}{-0.4em}


%\footnotesize

\bibliographystyle{apalike}
\bibliography{biblio}
%\end{multicols}



\end{document}
