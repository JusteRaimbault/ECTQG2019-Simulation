\documentclass[11pt]{article}
\input{header.tex}
\begin{document}

\title{\vspace{-1cm}Fostering the use of methods for geosimulation models sensitivity analysis and validation
\\\medskip
\textit{ECTQG 2019 - Abstract proposal}
}
\author{\noun{R. Reuillon}$^{1}$, \noun{M. Leclaire}$^{1}$, \noun{J. Raimbault}$^{1,\ast}$, \noun{H. Arduin}$^2$, \noun{P. Chapron}$^3$, \noun{G. Cherel}$^1$,\\
 \noun{E. Delay}$^4$, \noun{F. Lavall{\'e}e}$^5$, \noun{J. Passerat}$^6$, \noun{P. Peigne}$^7$, \noun{J. Perret}$^8$, \noun{S. Rey-Coyrehourcq}$^9$\medskip\\
$^1$ UPS CNRS 3611 ISC-PIF\\
$^2$ \\
$^3$ COGIT, IGN\\
$^4$ \\
$^5$ IRSTEA \\
$^6$ \\
$^7$ \\
$^8$ Univ. Paris-Est, LaSTIG STRUDEL, IGN, ENSG\\
$^9$ UMR CNRS 6266 IDEES
\medskip\\
$^{\ast}$\texttt{juste.raimbault@iscpif.fr}
}
\date{}

\maketitle

\justify

\pagenumbering{gobble}


\textbf{Keywords: }\textit{Validation of simulation models; Multi-modeling; Model calibration; Sensitivity analysis; Incremental modeling; Pedagogy of simulation models}

\medskip

Recent years have witnessed a significant development in methods to explore, validate, calibrate and optimize geosimulation models. These methods and tools remain however underused by simulation communities, despite a democratization in the accessibility to high performance computation facilities. The OpenMOLE model exploration software ~\citep{reuillon2013openmole} is one of the reliable approaches fully dedicated to promote these technics. This presentation proposes a feedback on the recent initiative of a researcher school in model validation, focused around models and practices linked to the OpenMOLE platform. We present the iterative exploration and validation protocol developed during the school, with methods of increasing refinement deployed on a toy geosimulation model (spatialized prey-predator model for a zombie infection with multi-modeling paradigms to include diverse processes for agent behavior). We illustrate first classical sensitivity analysis methods (stochasticity, design of experiment, global sensitivity indices), and then specific methods to study the sensitivity to spatial configuration, evolutionary computation methods for calibration and diversity search, and Bayesian calibration methods. They are applied on diverse specific submodels, highligting specific mechanisms of the model, in order to answer associated thematic questions. We also illustrate the comparison with competing model ontologies by calibrating a SIR model on data generated by the simulation model. We finally synthesize lessons learnt in a final challenge part of the school consisting in an autonomous exploration by participants of a new model instance unknown to them, including the problematization of a thematic question and the application of appropriated validation methods. This experiment both introduce a broad overview of new model geosimulation methods, and suggests ways to disseminate these into the modeling communities through similar pedagogical implementations.





%%%%%%%%%%%%%%%%%%%%
%% Biblio
%%%%%%%%%%%%%%%%%%%%
%\tiny

%\begin{multicols}{2}

%\setstretch{0.3}
%\setlength{\parskip}{-0.4em}


%\footnotesize

\bibliographystyle{apalike}
\bibliography{biblio}
%\end{multicols}



\end{document}
